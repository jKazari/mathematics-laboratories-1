\documentclass{book}
\usepackage[utf8]{inputenc}
\usepackage{polski}
\usepackage{XCharter}
\usepackage[all]{nowidow}
\usepackage{setspace}
\onehalfspacing
\usepackage{geometry}
\geometry{
 a4paper,
 left=30mm,
 right=30mm,
 top=30mm,
 bottom=30mm
}

\author{\Huge{Bolesław Prus}}
\title{\Huge{Lalka. Tom I}}
\date{}

\begin{document}

\maketitle

\tableofcontents

\chapter{Jak wygląda firma J. Mincel i S. Wokulski przez szkło butelek?}

W początkach roku 1878, kiedy świat polityczny zajmował się pokojem san-stefańskim, wyborem nowego papieża albo szansami europejskiej wojny, warszawscy kupcy tudzież inteligencja pewnej okolicy Krakowskiego Przedmieścia niemniej gorąco interesowała się przyszłością galanteryjnego sklepu pod firmą J. Mincel i S. Wokulski.

W renomowanej jadłodajni, gdzie na wieczorną przekąskę zbierali się właściciele składów bielizny i składów win, fabrykanci powozów i kapeluszy, poważni ojcowie rodzin, utrzymujący się z własnych funduszów, i posiadacze kamienic bez zajęcia, równie dużo mówiono o uzbrojeniach Anglii, jak o firmie J. Mincel i S. Wokulski. Zatopieni w kłębach dymu cygar i pochyleni nad butelkami z ciemnego szkła obywatele tej dzielnicy, jedni zakładali się o wygranę lub przegranę Anglii, drudzy o bankructwo Wokulskiego; jedni nazywali geniuszem Bismarcka, drudzy — awanturnikiem Wokulskiego; jedni krytykowali postępowanie prezydenta MacMahona, inni twierdzili, że Wokulski jest zdecydowanym wariatem, jeżeli nie czymś gorszym…

Pan Deklewski, fabrykant powozów, który majątek i stanowisko zawdzięczał wytrwałej pracy w jednym fachu, tudzież radca Węgrowicz, który od dwudziestu lat był członkiem-opiekunem jednego i tego samego Towarzystwa Dobroczynności, znali S. Wokulskiego najdawniej i najgłośniej przepowiadali mu ruinę. — Na ruinie bowiem i niewypłacalności — mówił pan Deklewski — musi skończyć człowiek, który nie pilnuje się jednego fachu i nie umie uszanować darów łaskawej fortuny. — Zaś radca Węgrowicz, po każdej również głębokiej sentencji swego przyjaciela, dodawał:

— Wariat! wariat!… Awanturnik!… Józiu, przynieś no jeszcze piwa. A która to butelka?

— Szósta, panie radco. Służę piorunem!… — odpowiadał Józio.

— Już szósta?… Jak ten czas leci!… Wariat! wariat! — mruczał radca Węgrowicz.

Dla osób posilających się w tej co radca jadłodajni, dla jej właściciela, subiektów i chłopców przyczyny klęsk mających paść na S. Wokulskiego i jego sklep galanteryjny były tak jasne, jak gazowe płomyki oświetlające zakład. Przyczyny te tkwiły w niespokojnym charakterze, w awanturniczym życiu, zresztą w najświeższym postępku człowieka, który mając w ręku pewny kawałek chleba i możność uczęszczania do tej oto tak przyzwoitej restauracji, dobrowolnie wyrzekł się restauracji, sklep zostawił na Opatrzności boskiej, a sam z całą gotówką odziedziczoną po żonie pojechał na turecką wojnę robić majątek.

— A może go i zrobi… Dostawy dla wojska to gruby interes — wtrącił pan Szprot, ajent handlowy, który bywał tu rzadkim gościem.

— Nic nie zrobi — odparł pan Deklewski — a tymczasem porządny sklep diabli wezmą. Na dostawach bogacą się tylko Żydzi i Niemcy; nasi do tego nie mają głowy.

— A może Wokulski ma głowę?

— Wariat! wariat!… — mruknął radca. — Podaj no, Józiu, piwa. Która to?…

— Siódma buteleczka, panie radco. Służę piorunem.

— Już siódma?… Jak ten czas leci, jak ten czas leci…

Ajent handlowy, który z obowiązków stanowiska potrzebował mieć o kupcach wiadomości wszechstronne i wyczerpujące, przeniósł swoją butelkę i szklankę do stołu radcy i topiąc słodkie wejrzenie w jego załzawionych oczach, spytał zniżonym głosem:

— Przepraszam, ale… Dlaczego pan radca nazywa Wokulskiego wariatem?… Może mogę służyć cygarkiem… Ja trochę znam Wokulskiego. Zawsze wydawał mi się człowiekiem skrytym i dumnym. W kupcu skrytość jest wielką zaletą, duma wadą. Ale żeby Wokulski zdradzał skłonności do wariacji, tegom nie spostrzegł.

Radca przyjął cygaro bez szczególnych oznak wdzięczności. Jego rumiana twarz, otoczona pękami siwych włosów nad czołem, na brodzie i na policzkach, była w tej chwili podobna do krwawnika oprawionego w srebro.

— Nazywam go — odparł, powoli ogryzając i zapalając cygaro — nazywam go wariatem, gdyż go znam lat… Zaczekaj pan… Piętnaście… siedemnaście… osiemnaście… Było to w roku 1860… Jadaliśmy wtedy u Hopfera. Znałeś pan Hopfera?…


— Phi…

— Otóż Wokulski był wtedy u Hopfera subiektem i miał już ze dwadzieścia parę lat.

— W handlu win i delikatesów?

— Tak. I jak dziś Józio, tak on wówczas podawał mi piwo, zrazy nelsońskie…

— I z tej branży przerzucił się do galanterii? — wtrącił ajent.

— Zaczekaj pan — przerwał radca. — Przerzucił się, ale nie do galanterii, tylko do Szkoły Przygotowawczej, a potem do Szkoły Głównej, rozumie pan?… Zachciało mu się być uczonym!…

Ajent począł chwiać głową w sposób oznaczający zdziwienie.

— Istna heca — rzekł. — I skąd mu to przyszło?

— No, skąd! Zwyczajnie — stosunki z Akademią Medyczną, ze Szkołą Sztuk Pięknych… Wtedy wszystkim paliło się we łbach, a on nie chciał być gorszym od innych. W dzień służył gościom przy bufecie i prowadził rachunki, a w nocy uczył się…

— Licha musiała to być usługa.

— Taka jak innych — odparł radca, niechętnie machając ręką. — Tylko że przy posłudze był, bestia, niemiły; na najniewinniejsze słówko marszczył się jak zbój… Rozumie się, używaliśmy na nim, co wlazło, a on najgorzej gniewał się, jeżeli nazwał go kto „panem konsyliarzem”. Raz tak zwymyślał gościa, że mało obaj nie porwali się za czuby.

— Naturalnie, handel cierpiał na tym.

— Wcale nie! Bo kiedy po Warszawie rozeszła się wieść, że subiekt Hopfera chce wstąpić do Szkoły Przygotowawczej, tłumy zaczęły tam przychodzić na śniadanie. Osobliwie roiła się studenteria.

— I poszedł też do Szkoły Przygotowawczej?

— Poszedł i nawet zdał egzamin do Szkoły Głównej. No, ale co pan powiesz — ciągnął radca uderzając ajenta w kolano — że zamiast wytrwać przy nauce do końca, niespełna w rok rzucił szkołę…



— Cóż robił?

— Otóż, co… Gotował wraz z innymi piwo, które do dziś dnia pijemy, i sam w rezultacie oparł się aż gdzieś koło Irkucka.

— Heca, panie! — westchnął ajent handlowy.

— Nie koniec na tym… W roku 1870 wrócił do Warszawy z niewielkim fundusikiem. Przez pół roku szukał zajęcia, z daleka omijając handle korzenne, których po dziś dzień nienawidzi, aż nareszcie przy protekcji swego dzisiejszego dysponenta, Rzeckiego, wkręcił się do sklepu Minclowej, która akurat została wdową, i w rok potem ożenił się z babą grubo starszą od niego.

— To nie było głupie — wtrącił ajent.

— Zapewne. Jednym zamachem zdobył sobie byt i warsztat, na którym mógł spokojnie pracować do końca życia. Ale też miał on krzyż Pański z babą!

— One to umieją…

— Jeszcze jak! — prawił radca. — Patrz pan jednakże, co to znaczy szczęście. Półtora roku temu baba objadła się czegoś i umarła, a Wokulski po czteroletniej katordze został wolny jak ptaszek, z zasobnym sklepem i trzydziestu tysiącami rubli w gotowiźnie, na którą pracowały dwa pokolenia Minclów.

— Ma szczęście.

— Miał — poprawił radca — ale go nie uszanował. Inny na jego miejscu ożeniłby się z jaką uczciwą panienką i żyłby w dostatkach; bo co to, panie, dziś znaczy sklep z reputacją i w doskonałym punkcie!… Ten jednak, wariat, rzucił wszystko i pojechał robić interesa na wojnie. Milionów mu się zachciało czy kiego diabła.

— Może je będzie miał — odezwał się ajent.

— Ehe! — żachnął się radca. — Daj no, Józiu, piwa. Myślisz pan, że w Turcji znajdzie jeszcze bogatszą babę aniżeli nieboszczka Minclowa?… Józiu!…

— Służę piorunem!… Jedzie ósma…

— Ósma? — powtórzył radca — to być nie może. Zaraz… Przedtem była szósta, potem siódma… — mruczał zasłaniając twarz dłonią. — Może być, że ósma. Jak ten czas leci!…

Mimo posępne wróżby ludzi trzeźwo patrzących na rzeczy, sklep galanteryjny pod firmą J. Mincel i S. Wokulski nie tylko nie upadł, ale nawet robił dobre interesa. Publiczność zaciekawiona pogłoskami o bankructwie coraz liczniej odwiedzała magazyn, od chwili zaś kiedy Wokulski opuścił Warszawę, zaczęli zgłaszać się po towary kupcy rosyjscy. Zamówienia mnożyły się, kredyt za granicą istniał, weksle były płacone regularnie, a sklep roił się gośćmi, którym ledwo mogli wydołać trzej subiekci: jeden mizerny blondyn, wyglądający, jakby co godzinę umierał na suchoty, drugi szatyn z brodą filozofa, a ruchami księcia i trzeci elegant, który nosił zabójcze dla płci pięknej wąsiki, pachnąc przy tym jak laboratorium chemiczne.

Ani jednak ciekawość ogółu, ani fizyczne i duchowe zalety trzech subiektów, ani nawet ustalona reputacja sklepu może nie uchroniłyby go od upadku, gdyby nie zawiadował nim czterdziestoletni pracownik firmy, przyjaciel i zastępca Wokulskiego, pan Ignacy Rzecki.

\chapter{Rządy starego subiekta}

Pan Ignacy od dwudziestu pięciu lat mieszkał w pokoiku przy sklepie. W ciągu tego czasu sklep zmieniał właścicieli i podłogę, szafy i szyby w oknach, zakres swojej działalności i subiektów; ale pokój pana Rzeckiego pozostał zawsze taki sam. Było w nim to samo smutne okno, wychodzące na to samo podwórze, z tą samą kratą, na której szczeblach zwieszała się, być może, ćwierćwiekowa pajęczyna, a z pewnością ćwierćwiekowa firanka, niegdyś zielona, obecnie wypłowiała z tęsknoty za słońcem.

Pod oknem stał ten sam czarny stół obity suknem, także niegdyś zielonym, dziś tylko poplamionym. Na nim wielki czarny kałamarz wraz z wielką czarną piaseczniczką, przymocowaną do tej samej podstawki — para mosiężnych lichtarzy do świec łojowych, których już nikt nie palił, i stalowe szczypce, którymi już nikt nie obcinał knotów. Żelazne łóżko z bardzo cienkim materacem, nad nim nigdy nie używana dubeltówka, pod nim pudło z gitarą, przypominające dziecinną trumienkę, wąska kanapka obita skórą, dwa krzesła również skórą obite, duża blaszana miednica i mała szafa ciemnowiśniowej barwy stanowiły umeblowanie pokoju, który, ze względu na swoją długość i mrok w nim panujący, zdawał się być podobniejszym do grobu aniżeli do mieszkania.

Równie jak pokój, nie zmieniły się od ćwierć wieku zwyczaje pana Ignacego.

Rano budził się zawsze o szóstej; przez chwilę słuchał, czy idzie leżący na krześle zegarek, i spoglądał na skazówki, które tworzyły jedną linię prostą. Chciał wstać spokojnie, bez awantur; ale że chłodne nogi i nieco zesztywniałe ręce nie okazywały się dość uległymi jego woli, więc zrywał się, nagle wyskakiwał na środek pokoju i rzuciwszy na łóżko szlafmycę, biegł pod piec do wielkiej miednicy, w której mył się od stóp do głów, rżąc i parskając jak wiekowy rumak szlachetnej krwi, któremu przypomniał się wyścig.

Podczas obrządku wycierania się kosmatymi ręcznikami, z upodobaniem patrzył na swoje chude łydki i zarośnięte piersi, mrucząc:

„No, przecie nabieram ciała.”

W tym samym czasie zeskakiwał z kanapki jego stary pudel Ir z wybitym okiem i mocno otrząsnąwszy się, zapewne z resztek snu, skrobał do drzwi, za którymi rozlegało się pracowite dmuchanie w samowar. Pan Rzecki, wciąż ubierając się z pośpiechem, wypuszczał psa, mówił dzień dobry służącemu, wydobywał z szafy imbryk, mylił się przy zapinaniu mankietów, biegł na podwórze zobaczyć stan pogody, parzył się gorącą herbatą, czesał się nie patrząc w lustro i o wpół do siódmej był gotów.

Obejrzawszy się, czy ma krawat na szyi, a zegarek i portmonetkę w kieszeniach, pan Ignacy wydobywał ze stolika wielki klucz i trochę zgarbiony, uroczyście otwierał tylne drzwi sklepu obite żelazną blachą. Wchodzili tam obaj ze służącym, zapalali parę płomyków gazu i podczas gdy służący zamiatał podłogę, pan Ignacy odczytywał przez binokle ze swego notatnika rozkład zajęć na dzień dzisiejszy.

„Oddać w banku osiemset rubli, aha… Do Lublina wysłać trzy albumy, tuzin portmonetek… Właśnie!… Do Wiednia przekaz na tysiąc dwieście guldenów… Z kolei odebrać transport… Zmonitować rymarza za nieodesłanie walizek… Bagatela!… Napisać list do Stasia… Bagatela…”

Skończywszy czytać, zapalał jeszcze kilka płomieni i przy ich blasku robił przegląd towarów w gablotkach i szafach.

„Spinki, szpilki, portmonety… dobrze… Rękawiczki, wachlarze, krawaty… tak jest… Laski, parasole, sakwojaże… A tu — albumy, neseserki… Szafirowy wczoraj sprzedano, naturalnie!… Lichtarze, kałamarze, przyciski… Porcelana… Ciekawym, dlaczego ten wazon odwrócili?… Z pewnością… Nie, nie uszkodzony… Lalki z włosami, teatr, karuzel… Trzeba na jutro postawić w oknie karuzel, bo już fontanna spowszedniała. Bagatela!… Ósma dochodzi… Założyłbym się, że Klejn będzie pierwszy, a Mraczewski ostatni. Naturalnie… Poznał się z jakąś guwernantką i już jej kupił neseserkę na rachunek i z rabatem… Rozumie się… Byle nie zaczął kupować bez rabatu i bez rachunku…”

Tak mruczał i chodził po sklepie przygarbiony, z rękoma w kieszeniach, a za nim jego pudel. Pan od czasu do czasu zatrzymywał się i oglądał jakiś przedmiot, pies przysiadał na podłodze i skrobał tylną nogą gęste kudły, a rzędem ustawione w szafie lalki małe, średnie i duże, brunetki i blondynki, przypatrywały się im martwymi oczami.

Drzwi od sieni skrzypnęły i ukazał się pan Klejn, mizerny subiekt, ze smutnym uśmiechem na posiniałych ustach.

— A co, byłem pewny, że pan przyjdziesz pierwszy. Dzień dobry — rzekł pan Ignacy. — Paweł! gaś światło i otwieraj sklep.

Służący wbiegł ciężkim kłusem i zakręcił gaz. Po chwili rozległo się zgrzytanie ryglów, szczękanie sztab i do sklepu wszedł dzień, jedyny gość, który nigdy nie zawodzi kupca. Rzecki usiadł przy kantorku pod oknem, Klejn stanął na zwykłym miejscu przy porcelanie.

— Pryncypał jeszcze nie wraca, nie miał pan listu? — spytał Klejn.

— Spodziewam się go w połowie marca, najdalej za miesiąc.

— Jeżeli go nie zatrzyma nowa wojna.

— Staś… Pan Wokulski — poprawił się Rzecki — pisze mi, że wojny nie będzie.

— Kursa jednak spadają, a przed chwilą czytałem, że flota angielska wpłynęła na Dardanele.

— To nic, wojny nie będzie. Zresztą — westchnął pan Ignacy — co nas obchodzi wojna, w której nie przyjmie udziału Bonaparte.

— Bonapartowie skończyli już karierę.

— Doprawdy?… — uśmiechnął się ironicznie pan Ignacy. — A na czyjąż korzyść MacMahon z Ducrotem układali w styczniu zamach stanu?… Wierz mi, panie Klejn, bonapartyzm to potęga!…

— Jest większa od niej.

— Jaka? — oburzył się pan Ignacy. — Może republika z Gambettą?… Może Bismarck?…

— Socjalizm… — szepnął mizerny subiekt kryjąc się za porcelanę.

Pan Ignacy mocniej zasadził binokle i podniósł się na swym fotelu, jakby pragnąc jednym zamachem obalić nową teorię, która przeciwstawiała się jego poglądom, lecz poplątało mu szyki wejście drugiego subiekta z brodą.

— A, moje uszanowanie panu Lisieckiemu! — zwrócił się do przybyłego. — Zimny dzień mamy, prawda? Która też godzina w mieście, bo mój zegarek musi się spieszyć. Jeszcze chyba nie ma kwadransa na dziewiątą?…

— Także koncept!… Pański zegarek zawsze spieszy się z rana, a późni wieczorem — odparł cierpko Lisiecki ocierając szronem pokryte wąsy.

— Założę się, żeś pan był wczoraj na preferansie.

— Ma się wiedzieć. Cóż pan myślisz, że mi na całą dobę wystarczy widok waszych galanterii i pańskiej siwizny?

— No, mój panie, wolę być trochę szpakowatym aniżeli łysym — oburzył się pan Ignacy.

— Koncept!… — syknął pan Lisiecki. — Moja łysina, jeżeli ją kto dojrzy, jest smutnym dziedzictwem rodu, ale pańska siwizna i gderliwy charakter są owocami starości, którą… chciałbym szanować.

Do sklepu wszedł pierwszy gość: kobieta ubrana w salopę i chustkę na głowie, żądająca mosiężnej spluwaczki… Pan Ignacy bardzo nisko ukłonił się jej i ofiarował krzesło, a pan Lisiecki zniknął za szafami i wróciwszy po chwili doręczył interesantce ruchem pełnym godności żądany przedmiot. Potem zapisał cenę spluwaczki na kartce, podał ją przez ramię Rzeckiemu i poszedł za gablotkę z miną bankiera, który złożył na cel dobroczynny kilka tysięcy rubli.

Spór o siwiznę i łysinę był zażegnany.

Dopiero około dziewiątej wszedł, a raczej wpadł do sklepu pan Mraczewski, piękny, dwudziestokilkoletni blondynek, z oczyma jak gwiazdy, ustami jak korale, z wąsikami jak zatrute sztylety. Wbiegł ciągnąc za sobą od progu smugę woni i zawołał:

— Słowo honoru daję, że musi już być wpół do dziesiątej. Letkiewicz jestem, gałgan jestem, no — podły jestem, ale cóż zrobię, kiedy matka mi zachorowała i musiałem szukać doktora. Byłem u sześciu…

— Czy u tych, którym dajesz pan neseserki? — spytał Lisiecki.

— Neseserki?… Nie. Nasz doktór nie przyjąłby nawet szpilki. Zacny człowiek… Prawda, panie Rzecki, że już jest wpół do dziesiątej? Stanął mi zegarek.


— Dochodzi dziewiąta… — odparł ze szczególnym naciskiem pan Ignacy.

— Dopiero dziewiąta?… No, kto by myślał! A tak projektowałem sobie, że dziś przyjdę do sklepu pierwszy, wcześniej od pana Klejna…

— Ażeby wyjść przed ósmą — wtrącił pan Lisiecki.

Mraczewski utkwił w nim błękitne oczy, w których malowało się najwyższe zdumienie.

— Pan skąd wie?… — odparł. — No, słowo honoru daję, że ten człowiek ma zmysł proroczy! Właśnie dziś, słowo honoru… muszę być na mieście przed siódmą, choćbym umarł, choćbym… miał podać się do dymisji…

— Niech pan od tego zacznie — wybuchnął Rzecki — a będzie pan wolny przed jedynastą, nawet w tej chwili, panie Mraczewski. Pan powinieneś być hrabią, nie kupcem, i dziwię się, że pan od razu nie wstąpił do tamtego fachu, przy którym zawsze ma się czas, panie Mraczewski. Naturalnie!

— No, i pan w jego wieku latałeś za spódniczkami — odezwał się Lisiecki. — Co tu bawić się w morały.

— Nigdy nie latałem! — krzyknął Rzecki uderzając pięścią w kantorek.

— Przynajmniej raz wygadał się, że całe życie jest niedołęgą — mruknął Lisiecki do Klejna, który uśmiechał się podnosząc jednocześnie brwi bardzo wysoko.

Do sklepu wszedł drugi gość i zażądał kaloszy. Naprzeciw niego wysunął się Mraczewski.

— Kaloszyków żąda szanowny pan? Który numerek, jeżeli wolno spytać? Ach, szanowny pan zapewne nie pamięta! Nie każdy ma czas myśleć o numerze swoich kaloszy, to należy do nas. Szanowny pan pozwoli, że przymierzymy?… Szanowny pan raczy zająć miejsce na taburecie. Paweł! przynieś ręcznik, zdejm panu kalosze i wytrzyj obuwie…

Wbiegł Paweł ze ścierką i rzucił się do nóg przybyłemu.

— Ależ, panie, ależ przepraszam!… — tłomaczył się odurzony gość.

— Bardzo prosimy — mówił prędko Mraczewski — to nasz obowiązek. Zdaje mi się, że te będą dobre — ciągnął podając parę sczepionych nitką kaloszy. — Doskonałe, pysznie wyglądają; szanowny pan ma tak normalną nogę, że niepodobna mylić się co do numeru. Szanowny pan życzy sobie zapewne literki; jakie mają być literki?…

— L. P. — mruknął gość czując, że tonie w bystrym potoku wymowy grzecznego subiekta.

— Panie Lisiecki, panie Klejn, przybijcie z łaski swojej literki. Szanowny pan każe zawinąć dawne kalosze? Paweł! wytrzyj kalosze i okręć w bibułę. A może szanowny pan nie życzy sobie dźwigać zbytecznego ciężaru? Paweł! rzuć kalosze do paki… Należy się dwa ruble kopiejek pięćdziesiąt… Kaloszy z literkami nikt szanownemu panu nie zamieni, a to przykra rzecz znaleźć w miejsce nowych artykułów dziurawe graty… Dwa ruble pięćdziesiąt kopiejek do kasy, z tą karteczką. Panie kasjerze, pięćdziesiąt kopiejek reszty dla szanownego pana…

Nim gość oprzytomniał, ubrano go w kalosze, wydano resztę i wśród niskich ukłonów odprowadzono do drzwi. Interesant stał przez chwilę na ulicy, bezmyślnie patrząc w szybę, spoza której Mraczewski darzył go słodkim uśmiechem i ognistymi spojrzeniami. Wreszcie machnął ręką i poszedł dalej, może myśląc, że w innym sklepie kalosze bez literek kosztowałyby go dziesięć złotych.

Pan Ignacy zwrócił się do Lisieckiego i kiwał głową w sposób oznaczający podziw i zadowolenie. Mraczewski dostrzegł ten ruch kątem oka i podbiegłszy do Lisieckiego, rzekł półgłosem:

— Niech no pan patrzy, czy nasz stary nie jest podobny z profilu do Napoleona III? Nos… wąs… hiszpanka…

— Do Napoleona, kiedy chorował na kamień — odparł Lisiecki.

Na ten dowcip pan Ignacy skrzywił się z niesmakiem. Swoją drogą Mraczewski dostał urlop przed siódmą wieczorem, a w parę dni później w prywatnym katalogu Rzeckiego otrzymał notatkę:

„Był na Hugonotach w ósmym rzędzie krzeseł z niejaką Matyldą…???”

Na pociechę mógłby sobie powiedzieć, że w tym samym katalogu równie posiadają notatki dwaj inni jego koledzy, a także inkasent, posłańcy, nawet — służący Paweł. Skąd Rzecki znał podobne szczegóły z życia swych współpracowników? Jest to tajemnica, z którą przed nikim się nie zwierzał.

Około pierwszej w południe pan Ignacy zdawszy kasę Lisieckiemu, któremu pomimo ciągłych sporów ufał najbardziej, wymykał się do swego pokoiku, ażeby zjeść obiad przyniesiony z restauracji. Współcześnie z nim wychodził Klejn i wracał do sklepu o drugiej; potem obaj z Rzeckim zostawali w sklepie, a Lisiecki i Mraczewski szli na obiad. O trzeciej znowu wszyscy byli na miejscu.

O ósmej wieczór zamykano sklep; subiekci rozchodzili się i zostawał tylko Rzecki. Robił dzienny rachunek, sprawdzał kasę, układał plan czynności na jutro i przypominał sobie: czy zrobiono wszystko, co wypadało na dziś? Każdą zaniedbaną sprawę opłacał długą bezsennością i smętnymi marzeniami na temat ruiny sklepu, stanowczego upadku Napoleonidów i tego, że wszystkie nadzieje, jakie miał w życiu, były tylko głupstwem.

„Nic nie będzie! Giniemy bez ratunku” — wzdychał przewracając się na twardej pościeli.

Jeżeli dzień udał się dobrze, pan Ignacy był kontent. Wówczas przed snem czytał historię konsulatu i cesarstwa albo wycinki z gazet opisujących wojnę włoską z roku 1859, albo też, co trafiało się rzadziej, wydobywał spod łóżka gitarę i grał na niej Marsza Rakoczego przyśpiewując wątpliwej wartości tenorem.

Potem śniły mu się obszerne węgierskie równiny, granatowe i białe linie wojsk, przysłoniętych chmurą dymu… Nazajutrz miewał posępny humor i skarżył się na ból głowy.

Do przyjemniejszych dni należała u niego niedziela; wówczas bowiem obmyślał i wykonywał plany wystaw okiennych na cały tydzień.

W jego pojęciu okna nie tylko streszczały zasoby sklepu, ale jeszcze powinny były zwracać uwagę przechodniów bądź najmodniejszym towarem, bądź pięknym ułożeniem, bądź figlem. Prawe okno przeznaczone dla galanteryj zbytkownych mieściło zwykle jakiś brąz, porcelanową wazę, całą zastawę buduarowego stolika, dokoła których ustawiały się albumy, lichtarze, portmonety, wachlarze, w towarzystwie lasek, parasoli i niezliczonej ilości drobnych, a eleganckich przedmiotów. W lewym znowu oknie, napełnionym okazami krawatów, rękawiczek, kaloszy i perfum, miejsce środkowe zajmowały zabawki, najczęściej poruszające się.

Niekiedy podczas tych samotnych zajęć w starym subiekcie budziło się dziecko. Wydobywał wtedy i ustawiał na stole wszystkie mechaniczne cacka. Był tam niedźwiedź wdrapujący się na słup, był piejący kogut, mysz, która biegała, pociąg, który toczył się po szynach, cyrkowy pajac, który cwałował na koniu, dźwigając drugiego pajaca, i kilka par, które tańczyły walca przy dźwiękach niewyraźnej muzyki. Wszystkie te figury pan Ignacy nakręcał i jednocześnie puszczał w ruch. A gdy kogut zaczął piać łopocząc sztywnymi skrzydłami, gdy tańczyły martwe pary, co chwilę potykając się i zatrzymując, gdy ołowiani pasażerowie pociągu, jadącego bez celu, zaczęli przypatrywać mu się ze zdziwieniem i gdy cały ten świat lalek, przy drgającym świetle gazu, nabrał jakiegoś fantastycznego życia, stary subiekt podparłszy się łokciami śmiał się cicho i mruczał:

— Hi! hi! hi! dokąd wy jedziecie, podróżni?… Dlaczego narażasz kark, akrobato?… Co wam po uściskach, tancerze?… Wykręcą się sprężyny i pójdziecie na powrót do szafy. Głupstwo, wszystko głupstwo!… a wam, gdybyście myśleli, mogłoby się zdawać, że to jest coś wielkiego!…

Po takich i tym podobnych monologach szybko składał zabawki i rozdrażniony chodził po pustym sklepie, a za nim jego brudny pies.

„Głupstwo handel… głupstwo polityka… głupstwo podróż do Turcji… głupstwo całe życie, którego początku nie pamiętamy, a końca nie znamy… Gdzież prawda?…”

Ponieważ tego rodzaju zdania wypowiadał niekiedy głośno i publicznie, więc uważano go za bzika, a poważne damy, mające córki na wydaniu, nieraz mówiły.


— Oto do czego prowadzi mężczyznę starokawalerstwo!

Z domu pan Ignacy wychodził rzadko i na krótko i zwykle kręcił się po ulicach, na których mieszkali jego koledzy albo oficjaliści sklepu. Wówczas jego ciemnozielona algierka lub tabaczkowy surdut, popielate spodnie z czarnym lampasem i wypłowiały cylinder, nade wszystko zaś jego nieśmiałe zachowanie się zwracały powszechną uwagę. Pan Ignacy wiedział to i coraz bardziej zniechęcał się do spacerów. Wolał przy święcie kłaść się na łóżku i całymi godzinami patrzeć w swoje zakratowane okno, za którym widać było szary mur sąsiedniego domu, ozdobiony jednym jedynym, również zakratowanym oknem, gdzie czasami stał garnczek masła albo wisiały zwłoki zająca.

Lecz im mniej wychodził, tym częściej marzył o jakiejś dalekiej podróży na wieś lub za granicę. Coraz częściej spotykał we snach zielone pola i ciemne bory, po których błąkałby się, przypominając sobie młode czasy. Powoli zbudziła się w nim głucha tęsknota do tych krajobrazów, więc postanowił natychmiast po powrocie Wokulskiego wyjechać gdzieś na całe lato.

— Choć raz przed śmiercią, ale na kilka miesięcy — mówił kolegom, którzy nie wiadomo dlaczego uśmiechali się z tych projektów.

Dobrowolnie odcięty od natury i ludzi, utopiony w wartkim, ale ciasnym wirze sklepowych interesów, czuł coraz mocniej potrzebę wymiany myśli. A ponieważ jednym nie ufał, inni go nie chcieli słuchać, a Wokulskiego nie było, więc rozmawiał sam z sobą i — w największym sekrecie pisywał pamiętnik.

\chapter{Pamiętnik starego subiekta}

…Ze smutkiem od kilku lat uważam, że na świecie jest coraz mniej dobrych subiektów i rozumnych polityków, bo wszyscy stosują się do mody. Skromny subiekt co kwartał ubiera się w spodnie nowego fasonu, w coraz dziwniejszy kapelusz i coraz inaczej wykładany kołnierzyk. Podobnież dzisiejsi politycy co kwartał zmieniają wiarę: onegdaj wierzyli w Bismarcka, wczoraj w Gambettę, a dziś w Beaconsfielda, który niedawno był Żydkiem.

Już widać zapomniano, że w sklepie nie można stroić się w modne kołnierzyki, tylko je sprzedawać, bo w przeciwnym razie gościom zabraknie towaru, a sklepowi gości. Zaś polityki nie należy opierać na szczęśliwych osobach, tylko na wielkich dynastiach. Metternich był taki sławny jak Bismarck, a Palmerston sławniejszy od Beaconsfielda i — któż dziś o nich pamięta? Tymczasem ród Bonapartych trząsł Europą za Napoleona I, potem za Napoleona III, a i dzisiaj, choć niektórzy nazywają go bankrutem, wpływa na losy Francji przez wierne swoje sługi, MacMahona i Ducrota.

Zobaczycie, co jeszcze zrobi Napoleonek IV, który po cichu uczy się sztuki wojennej u Anglików! Ale o to mniejsza. W tej bowiem pisaninie chcę mówić nie o Bonapartym, ale o sobie, ażeby wiedziano, jakim sposobem tworzyli się dobrzy subiekci i choć nie uczeni, ale rozsądni politycy. Do takiego interesu nie trzeba akademii, lecz przykładu — w domu i w sklepie.

Ojciec mój był za młodu żołnierzem, a na starość woźnym w Komisji Spraw Wewnętrznych. Trzymał się prosto jak sztaba, miał nieduże faworyty i wąs do góry; szyję okręcał czarną chustką i nosił srebrny kolczyk w uchu.

Mieszkaliśmy na Starym Mieście z ciotką, która urzędnikom prała i łatała bieliznę. Mieliśmy na czwartym piętrze dwa pokoiki, gdzie niewiele było dostatków, ale dużo radości, przynajmniej dla mnie. W naszej izdebce najokazalszym sprzętem był stół, na którym ojciec powróciwszy z biura kleił koperty; u ciotki zaś pierwsze miejsce zajmowała balia. Pamiętam, że w pogodne dnie puszczałem na ulicy latawce, a w razie słoty wydmuchiwałem w izbie bańki mydlane.

Na ścianach u ciotki wisieli sami święci; ale jakkolwiek było ich sporo, nie dorównali jednak liczbą Napoleonom, którymi ojciec przyozdabiał swój pokój. Był tam jeden Napoleon w Egipcie, drugi pod Wagram, trzeci pod Austerlitz, czwarty pod Moskwą, piąty w dniu koronacji, szósty w apoteozie. Gdy zaś ciotka zgorszona tyloma świeckimi obrazami, zawiesiła na ścianie mosiężny krucyfiks, ojciec, ażeby — jak mówił — nie obrazić Napoleona, kupił sobie jego brązowe popiersie i także umieścił je nad łóżkiem.

— Zobaczysz, niedowiarku — lamentowała nieraz ciotka — że za te sztuki będą cię pławić w smole.

— I!… Nie da mi cesarz zrobić krzywdy — odpowiadał ojciec.

Często przychodzili do nas dawni koledzy ojca: pan Domański, także woźny, ale z Komisji Skarbu, i pan Raczek, który na Dunaju miał stragan z zieleniną. Prości to byli ludzie (nawet pan Domański trochę lubił anyżówkę), ale roztropni politycy. Wszyscy, nie wyłączając ciotki, twierdzili jak najbardziej stanowczo, że choć Napoleon I umarł w niewoli, ród Bonapartych jeszcze wypłynie. Po pierwszym Napoleonie znajdzie się jakiś drugi, a gdyby i ten źle skończył, przyjdzie następny, dopóki jeden po drugim nie uporządkują świata.

— Trzeba być zawsze gotowym na pierwszy odgłos! — mówił mój ojciec.

— Bo nie wiecie dnia ani godziny — dodawał pan Domański.

A pan Raczek, trzymając fajkę w ustach, na znak potwierdzenia pluł aż do pokoju ciotki.

— Napluj mi acan w balię, to ci dam!… — wołała ciotka.

— Może jejmość i dasz, ale ja nie wezmę — mruknął pan Raczek plując w stronę komina.

— U… cóż to za chamy te całe grenadierzyska! — gniewała się ciotka.

— Jejmości zawsze smakowali ułani. Wiem, wiem…

Później pan Raczek ożenił się z moją ciotką…

…Chcąc, ażebym zupełnie był gotów, gdy wybije godzina sprawiedliwości, ojciec sam pracował nad moją edukacją.

Nauczył mię czytać, pisać, kleić koperty, ale nade wszystko — musztrować się. Do musztry zapędzał mnie w bardzo wczesnym dzieciństwie, kiedy mi jeszcze zza pleców wyglądała koszula. Dobrze to pamiętam, gdyż ojciec komenderując: „Pół obrotu na prawo!” albo „Lewe ramię naprzód — marsz!…”, ciągnął mnie w odpowiednim kierunku za ogon tego ubrania.

Była to najdokładniej prowadzona nauka.

Nieraz w nocy budził mnie ojciec krzykiem: „Do broni!…”, musztrował pomimo wymyślań i łez ciotki i kończył zdaniem:

— Ignaś! zawsze bądź gotów, wisusie, bo nie wiemy dnia ani godziny… Pamiętaj, że Bonapartów Bóg zesłał, ażeby zrobili porządek na świecie, a dopóty nie będzie porządku ani sprawiedliwości, dopóki nie wypełni się testament cesarza.

Nie mogę powiedzieć, ażeby niezachwianą wiarę mego ojca w Bonapartych i sprawiedliwość podzielali dwaj jego koledzy. Nieraz pan Raczek, kiedy mu dokuczył ból w nodze, klnąc i stękając mówił:

— E! wiesz, stary, że już za długo czekamy na nowego Napoleona. Ja siwieć zaczynam i coraz gorzej podupadam, a jego jak nie było, tak i nie ma. Niedługo porobią się z nas dziady pod kościół, a Napoleon po to chyba przyjdzie, ażeby z nami śpiewać godzinki.

— Znajdzie młodych.

— Co za młodych! Lepsi z nich przed nami poszli w ziemię, a najmłodsi — diabła warci. Już są między nimi i tacy, co o Napoleonie nie słyszeli.

— Mój słyszał i zapamięta — odparł ojciec mrugając okiem w moją stronę.

Pan Domański jeszcze bardziej upadał na duchu.

— Świat idzie do gorszego — mówił trzęsąc głową. — Wikt coraz droższy, za kwaterę zabraliby ci całą pensję, a nawet co się tyczy anyżówki, i w tym jest szachrajstwo. Dawniej rozweseliłeś się kieliszkiem, dziś po szklance jesteś taki czczy, jakbyś się napił wody. Sam Napoleon nie doczekałby się sprawiedliwości!

A na to odpowiedział ojciec:


— Będzie sprawiedliwość, choćby i Napoleona nie stało. Ale i Napoleon się znajdzie.

— Nie wierzę — mruknął pan Raczek.

— A jak się znajdzie, to co?… — spytał ojciec.

— Nie doczekamy tego.

— Ja doczekam — odparł ojciec — a Ignaś doczeka jeszcze lepiej.

Już wówczas zdania mego ojca głęboko wyrzynały mi się w pamięci, ale dopiero późniejsze wypadki nadały im cudowny, nieomal proroczy charakter.

Około roku 1840 ojciec zaczął niedomagać. Czasami po parę dni nie wychodził do biura, a wreszcie na dobre legł w łóżku.

Pan Raczek odwiedzał go co dzień, a raz patrząc na jego chude ręce i wyżółkłe policzki szepnął:

— Hej! stary, już my chyba nie doczekamy się Napoleona.

Na co ojciec spokojnie odparł:

— Ja tam nie umrę, dopóki o nim nie usłyszę.

Pan Raczek pokiwał głową, a ciotka łzy otarła myśląc, że ojciec bredzi. Jak tu myśleć inaczej, jeżeli śmierć już kołatała do drzwi, a ojciec jeszcze wyglądał Napoleona…

Było już z nim bardzo źle, nawet przyjął ostatnie sakramenta, kiedy w parę dni później wbiegł do nas pan Raczek dziwnie wzburzony i stojąc na środku izby zawołał:

— A wiesz, stary, że znalazł się Napoleon?…

— Gdzie? — krzyknęła ciotka.

— Jużci we Francji.

Ojciec zerwał się, lecz znowu upadł na poduszki. Tylko wyciągnął do mnie rękę i patrząc wzrokiem, którego nie zapomnę, wyszeptał:

— Pamiętaj!… Wszystko pamiętaj…

Z tym umarł.

W późniejszym życiu przekonałem się, jak proroczymi były poglądy ojca. Wszyscy widzieliśmy drugą gwiazdę napoleońską, która obudziła Włochy i Węgry; a chociaż spadła pod Sedanem, nie wierzę w jej ostateczne zagaśnięcie. Co mi tam Bismarck, Gambetta albo Beaconsfield! Niesprawiedliwość dopóty będzie władać światem, dopóki nowy Napoleon nie urośnie.



W parę miesięcy po śmierci ojca pan Raczek i pan Domański wraz z ciotką Zuzanną zebrali się na radę: co ze mną począć? Pan Domański chciał mnie zabrać do swoich biur i powoli wypromować na urzędnika; ciotka zalecała rzemiosło, a pan Raczek zieleniarstwo. Lecz gdy zapytano mnie: do czego mam ochotę? odpowiedziałem, że do sklepu.

— Kto wie, czy to nie będzie najlepsze — zauważył pan Raczek. — A do jakiegoż byś chciał kupca?

— Do tego na Podwalu, co ma we drzwiach pałasz, a w oknie kozaka.

— Wiem — wtrąciła ciotka. — On chce do Mincla.

— Można spróbować — rzekł pan Domański. — Wszyscy przecież znamy Mincla.

Pan Raczek na znak zgody plunął aż w komin.

— Boże miłosierny — jęknęła ciotka — ten drab już chyba na mnie pluć zacznie, kiedy brata nie stało… Oj! nieszczęśliwa ja sierota!…

— Wielka rzecz! — odezwał się pan Raczek. — Wyjdź jejmość za mąż, to nie będziesz sierotą.

— A gdzież ja znajdę takiego głupiego, co by mnie wziął?

— Phi! może i ja bym się ożenił z jejmością, bo nie ma mnie kto smarować — mruknął pan Raczek, ciężko schylając się do ziemi, ażeby wypukać popiół z fajki. Ciotka rozpłakała się, a wtedy odezwał się pan Domański:

— Po co robić duże ceregiele. Jejmość nie masz opieki, on nie ma gospodyni; pobierzcie się i przygarnijcie Ignasia, a będziecie nawet mieli dziecko. Jeszcze tanie dziecko, bo Mincel da mu wikt i kwaterę, a wy tylko odzież.

— Hę?… — spytał pan Raczek patrząc na ciotkę.

— No, oddajcie pierwej chłopca do terminu, a potem… może się odważę — odparła ciotka. — Zawsze miałam przeczucie, że marnie skończę…

— To i jazda do Mincla! — rzekł pan Raczek podnosząc się z krzesełka. — Tylko jejmość nie zrób mi zawodu! — dodał grożąc ciotce pięścią.

Wyszli z panem Domańskim i może w półtorej godziny wrócili obaj mocno zarumienieni. Pan Raczek ledwie oddychał, a pan Domański z trudnością trzymał się na nogach, podobno z tego, że nasze schody były bardzo niewygodne.

— Cóż?… — spytała ciotka.

— Nowego Napoleona wsadzili do prochowni! — odpowiedział pan Domański.

— Nie do prochowni, tylko do fortecy. A–u… A–u… — dodał pan Raczek i rzucił czapkę na stół.

— Ale z chłopcem co?

— Jutro ma przyjść do Mincla z odzieniem i bielizną — odrzekł pan Domański. — Nie do fortecy A–u…, A–u… tylko do Ham–ham czy Cham… bo nawet nie wiem…

— Zwariowaliście, pijaki! — krzyknęła ciotka chwytając pana Raczka za ramię.

— Tylko bez poufałości! — oburzył się pan Raczek. — Po ślubie będzie poufałość, teraz… Ma przyjść do Mincla jutro z bielizną i odzieniem… Nieszczęsny Napoleonie!…

Ciotka wypchnęła za drzwi pana Raczka, potem pana Domańskiego i wyrzuciła za nimi czapkę.

— Precz mi stąd, pijaki!

— Wiwat Napoleon! — zawołał pan Raczek, a pan Domański zaczął śpiewać:

Przechodniu, gdy w tę stronę zwrócisz swoje oko,
Przybliż się i rozważaj ten napis głęboko…
Przybliż się i rozważaj ten napis głęboko. 
Głos jego stopniowo cichnął, jakby zagłębiając się w studni, potem umilkł na schodach, lecz znowu doleciał nas z ulicy. Po chwili zrobił się tam jakiś hałas, a gdy wyjrzałem oknem, zobaczyłem, że pana Raczka policjant prowadził do ratusza.

Takie to wypadki poprzedziły moje wejście do zawodu kupieckiego.

Sklep Mincla znałem od dawna, ponieważ ojciec wysyłał mnie do niego po papier, a ciotka po mydło. Zawsze biegłem tam z radosną ciekawością, ażeby napatrzeć się wiszącym za szybami zabawkom. O ile pamiętam, był tam w oknie duży kozak, który sam przez się skakał i machał rękoma, a we drzwiach — bęben, pałasz i skórzany koń z prawdziwym ogonem.

Wnętrze sklepu wyglądało jak duża piwnica, której końca nigdy nie mogłem dojrzeć z powodu ciemności. Wiem tylko, że po pieprz, kawę i liście bobkowe szło się na lewo do stołu, za którym stały ogromne szafy od sklepienia do podłogi napełnione szufladami. Papier zaś, atrament, talerze i szklanki sprzedawano przy stole na prawo, gdzie były szafy z szybami, a po mydło i krochmal szło się w głąb sklepu, gdzie było widać beczki i stosy pak drewnianych.

Nawet sklepienie było zajęte. Wisiały tam długie szeregi pęcherzy naładowanych gorczycą i farbami, ogromna lampa z daszkiem, która w zimie paliła się cały dzień, sieć pełna korków do butelek, wreszcie — wypchany krokodylek, długi może na półtora łokcia.

Właścicielem sklepu był Jan Mincel, starzec z rumianą twarzą i kosmykiem siwych włosów pod brodą. W każdej porze dnia siedział on pod oknem na fotelu obitym skórą, ubrany w niebieski barchanowy kaftan, biały fartuch i takąż szlafmycę. Przed nim na stole leżała wielka księga, w której notował dochód, a tuż nad jego głową wisiał pęk dyscyplin, przeznaczonych głównie na sprzedaż. Starzec odbierał pieniądze, zdawał gościom resztę, pisał w księdze, niekiedy drzemał, lecz pomimo tylu zajęć, z niepojętą uwagą czuwał nad biegiem handlu w całym sklepie. On także, dla uciechy przechodniów ulicznych, od czasu do czasu pociągał za sznurek skaczącego w oknie kozaka i on wreszcie, co mi się najmniej podobało, za rozmaite przestępstwa karcił nas jedną z pęka dyscyplin.

Mówię: nas, bo było nas trzech kandydatów do kary cielesnej: ja tudzież dwaj synowcy starego — Franc i Jan Minclowie.

Czujności pryncypała i jego biegłości w używaniu sarniej nogi doświadczyłem zaraz na trzeci dzień po wejściu do sklepu.

Franc odmierzył jakiejś kobiecie za dziesięć groszy rodzynków. Widząc, że jedno ziarno upadło na kontuar (stary miał w tej chwili oczy zamknięte), podniosłem je nieznacznie i zjadłem. Chciałem właśnie wyjąć pestkę, która wcisnęła mi się między zęby, gdy uczułem na plecach coś, jakby mocne dotknięcie rozpalonego żelaza.

— A, szelma! — wrzasnął stary Mincel i nim zdałem sobie sprawę z sytuacji, przeciągnął po mnie jeszcze parę razy dyscyplinę, od wierzchu głowy do podłogi.

Zwinąłem się w kłębek z bólu, lecz od tej pory nie śmiałem wziąć do ust niczego w sklepie. Migdały, rodzynki, nawet rożki miały dla mnie smak pieprzu.

Urządziwszy się ze mną w taki sposób, stary zawiesił dyscyplinę na pęku, wpisał rodzynki i z najdobroduszniejszą miną począł ciągnąć za sznurek kozaka. Patrząc na jego półuśmiechniętą twarz i przymrużone oczy, prawie nie mogłem wierzyć, że ten jowialny staruszek posiada taki zamach w ręku. I dopiero teraz spostrzegłem, że ów kozak widziany z wnętrza sklepu wydaje się mniej zabawnym niż od ulicy.

Sklep nasz był kolonialno-galanteryjno-mydlarski. Towary kolonialne wydawał gościom Franc Mincel, młodzieniec trzydziestokilkoletni, z rudą głową i zaspaną fizjognomią. Ten najczęściej dostawał dyscypliną od stryja, gdyż palił fajkę, późno wchodził za kontuar, wymykał się z domu po nocach, a nade wszystko niedbale ważył towar. Młodszy zaś, Jan Mincel, który zawiadywał galanterią i obok niezgrabnych ruchów odznaczał się łagodnością, był znowu bity za wykradanie kolorowego papieru i pisywanie na nim listów do panien.

Tylko August Katz, pracujący przy mydle, nie ulegał żadnym surowcowym upomnieniom. Mizerny ten człeczyna odznaczał się niezwykłą punktualnością. Najraniej przychodził do roboty, krajał mydło i ważył krochmal jak automat; jadł, co mu podano, w najciemniejszym kącie sklepu, prawie wstydząc się tego, że doświadcza ludzkich potrzeb. O dziesiątej wieczorem gdzieś znikał.

W tym otoczeniu upłynęło mi osiem lat, z których każdy dzień był podobny do wszystkich innych dni, jak kropla jesiennego deszczu do innych kropli jesiennego deszczu. Wstawałem rano o piątej, myłem się i zamiatałem sklep. O szóstej otwierałem główne drzwi tudzież okiennicę. W tej chwili gdzieś z ulicy zjawiał się August Katz, zdejmował surdut, kładł fartuch i milcząc stawał między beczką mydła szarego a kolumną ułożoną z cegiełek mydła żółtego. Potem drzwiami od podwórka wbiegał stary Mincel mrucząc: Morgen!, poprawiał szlafmycę, dobywał z szuflady księgę, wciskał się w fotel i parę razy ciągnął za sznurek kozaka. Dopiero po nim ukazywał się Jan Mincel i ucałowawszy stryja w rękę stawał za swoim kontuarem, na którym podczas lata łapał muchy, a w zimie kreślił palcem albo pięścią jakieś figury.

Franca zwykle sprowadzano do sklepu. Wchodził z oczyma zaspanymi, ziewający, obojętnie całował stryja w ramię i przez cały dzień skrobał się w głowę w sposób, który mógł oznaczać wielką senność lub wielkie zmartwienie. Prawie nie było ranka, ażeby stryj patrząc na jego manewry nie wykrzywiał mu się i nie pytał:

— No… a gdzie, ty szelma, latała?

Tymczasem na ulicy budził się szmer i za szybami sklepu coraz częściej przesuwali się przechodnie. To służąca, to drwal, jejmość w kapturze, to chłopak od szewca, to jegomość w rogatywce szli w jedną i drugą stronę jak figury w ruchomej panoramie. Środkiem ulicy toczyły się wozy, beczki, bryczki — tam i na powrót… Coraz więcej ludzi, coraz więcej wozów, aż nareszcie utworzył się jeden wielki potok uliczny, z którego co chwilę ktoś wpadał do nas za sprawunkiem.

— Pieprzu za trojaka…

— Proszę funt kawy…

— Niech pan da ryżu…

— Pół funta mydła…

— Za grosz liści bobkowych…

Stopniowo sklep zapełniał się po największej części służącymi i ubogo odzianymi jejmościami. Wtedy Franc Mincel krzywił się najwięcej: otwierał i zamykał szuflady, obwijał towar w tutki z szarej bibuły, wbiegał na drabinkę, znowu zwijał, robiąc to wszystko z żałosną miną człowieka, któremu nie pozwalają ziewnąć. W końcu zbierało się takie mnóstwo interesantów, że i Jan Mincel, i ja musieliśmy pomagać Francowi w sprzedaży.

Stary wciąż pisał i zdawał resztę, od czasu do czasu dotykając palcami swojej białej szlafmycy, której niebieski kutasik zwieszał mu się nad okiem. Czasem szarpnął kozaka, a niekiedy z szybkością błyskawicy zdejmował dyscyplinę i ćwiknął nią którego ze swych synowców. Nader rzadko mogłem zrozumieć: o co mu chodzi? synowcy bowiem niechętnie objaśniali mi przyczyny jego popędliwości.

Około ósmej napływ interesantów zmniejszał się. Wtedy w głębi sklepu ukazywała się gruba służąca z koszem bułek i kubkami (Franc odwracał się do niej tyłem), a za nią — matka naszego pryncypała, chuda staruszka w żółtej sukni, w ogromnym czepcu na głowie, z dzbankiem kawy w rękach. Ustawiwszy na stole swoje naczynie, staruszka odzywała się schrypniętym głosem:

— Gut Morgen, meine Kinder! Der Kaffee ist schon fertig…
I zaczynała rozlewać kawę w białe fajansowe kubki.

Wówczas zbliżał się do niej stary Mincel i całował ją w rękę mówiąc:

— Gut Morgen, meine Mutter!
Za co dostawał kubek kawy z trzema bułkami.

Potem przychodził Franc Mincel, Jan Mincel, August Katz, a na końcu ja. Każdy całował staruszkę w suchą rękę, porysowaną niebieskimi żyłami, każdy mówił:

— Gut Morgen, Grossmutter!
I otrzymywał należny mu kubek tudzież trzy bułki.

A gdyśmy z pośpiechem wypili naszą kawę, służąca zabierała pusty kosz i zamazane kubki, staruszka swój dzbanek i obie znikały.

Za oknem wciąż toczyły się wozy i płynął w obie strony potok ludzki, z którego co chwila odrywał się ktoś i wchodził do sklepu.

— Proszę krochmalu…

— Dać migdałów za dziesiątkę…

— Lukrecji za grosz…

— Szarego mydła…

Około południa zmniejszał się ruch za kontuarem towarów kolonialnych, a za to coraz częściej zjawiali się interesanci po stronie prawej sklepu, u Jana. Tu kupowano talerze, szklanki, żelazka, młynki, lalki, a niekiedy duże parasole, szafirowe lub pąsowe. Nabywcy, kobiety i mężczyźni, byli dobrze ubrani, rozsiadali się na krzesłach i kazali sobie pokazywać mnóstwo przedmiotów targując się i żądając coraz to nowych.

Pamiętam, że kiedy po lewej stronie sklepu męczyłem się bieganiną i zawijaniem towarów, po prawej — największe strapienie robiła mi myśl: czego ten a ten gość chce naprawdę i — czy co kupi? W rezultacie jednak i tutaj dużo się sprzedawało; nawet dzienny dochód z galanterii był kilka razy większy aniżeli z towarów kolonialnych i mydła.

Stary Mincel i w niedzielę bywał w sklepie. Rano modlił się, a około południa kazał mi przychodzić do siebie na pewien rodzaj lekcji.

— Sag mir — powiedz mi: was ist das? co to jest? Das ist Schublade — to jest szublada. Zobacz, co jest w te szublade. Es ist Zimmt — to jest cynamon. Do czego potrzebuje się cynamon? do zupe, do legumine potrzebuje się cynamon. Co to jest cynamon? Jest taki kora z jedne drzewo. Gdzie mieszka taki drzewo cynamon? W Indii mieszka taki drzewo. Patrz na globus — tu leży Indii. Daj mnie za dziesiątkę cynamon… O, du Spitzbub!… jak tobie dam dziesięć razy dyscyplin, ty będziesz wiedział, ile sprzedać za dziesięć groszy cynamon…

W ten sposób przechodziliśmy każdą szufladę w sklepie i historię każdego towaru. Gdy zaś Mincel nie był zmęczony, dyktował mi jeszcze zadania rachunkowe, kazał sumować księgi albo pisywać listy w interesach naszego sklepu.

Mincel był bardzo porządny, nie cierpiał kurzu, ścierał go z najdrobniejszych przedmiotów. Jednych tylko dyscyplin nigdy nie potrzebował okurzać dzięki swoim niedzielnym wykładom buchalterii, jeografii i towaroznawstwa.

Powoli, w ciągu paru lat, tak przywykliśmy do siebie, że stary Mincel nie mógł obejść się beze mnie, a ja nawet jego dyscypliny począłem uważać za coś, co należało do familijnych stosunków. Pamiętam, że nie mogłem utulić się z żalu, gdy raz zepsułem kosztowny samowar, a stary Mincel zamiast chwytać za dyscyplinę — odezwał się:

— Co ty zrobila, Ignac?… Co ty zrobila!…

Wolałbym dostać cięgi wszystkimi dyscyplinami, aniżeli znowu kiedy usłyszeć ten drżący głos i zobaczyć wylęknione spojrzenie pryncypała.

Obiady w dzień powszedni jadaliśmy w sklepie, naprzód dwaj młodzi Minclowie i August Katz, a następnie ja z pryncypałem. W czasie święta wszyscy zbieraliśmy się na górze i zasiadaliśmy do jednego stołu. Na każdą Wigilię Bożego Narodzenia Mincel dawał nam podarunki, a jego matka w największym sekrecie urządzała nam (i swemu synowi) choinkę. Wreszcie w pierwszym dniu miesiąca wszyscy dostawaliśmy pensję (ja brałem 10 złotych). Przy tej okazji każdy musiał wylegitymować się z porobionych oszczędności: ja, Katz, dwaj synowcy i służba. Nierobienie oszczędności, a raczej nieodkładanie co dzień choćby kilku groszy, było w oczach Mincla takim występkiem jak kradzież. Za mojej pamięci przewinęło się przez nasz sklep paru subiektów i kilku uczniów, których pryncypał dlatego tylko usunął, że nic sobie nie oszczędzili. Dzień, w którym się to wydało, był ostatnim ich pobytu. Nie pomogły obietnice, zaklęcia, całowania po rękach, nawet upadanie do nóg. Stary nie ruszył się z fotelu, nie patrzył na petentów, tylko wskazując palcem drzwi wymawiał jeden wyraz: fort! fort!… Zasada robienia oszczędności stała się już u niego chorobliwym dziwactwem.

Dobry ten człowiek miał jedną wadę, oto — nienawidził Napoleona. Sam nigdy o nim nie wspominał, lecz na dźwięk nazwiska Bonapartego dostawał jakby ataku wścieklizny; siniał na twarzy, pluł i wrzeszczał: szelma! szpitzbub! rozbójnik!…

Usłyszawszy pierwszy raz tak szkaradne wymysły nieomal straciłem przytomność. Chciałem coś hardego powiedzieć staremu i uciec do pana Raczka, który już ożenił się z moją ciotką. Nagle dostrzegłem, że Jan Mincel zasłoniwszy usta dłonią coś mruczy i robi miny do Katza. Wytężam słuch i — oto co mówi Jan:

— Baje stary, baje! Napoleon był chwat, choćby za to samo, że wygnał hyclów Szwabów. Nieprawda, Katz?

A August Katz zmrużył oczy i dalej krajał mydło.


Osłupiałem ze zdziwienia, lecz w tej chwili bardzo polubiłem Jana Mincla i Augusta Katza. Z czasem przekonałem się, że w naszym małym sklepie istnieją aż dwa wielkie stronnictwa, z których jedno, składające się ze starego Mincla i jego matki, bardzo lubiło Niemców, a drugie, złożone z młodych Minclów i Katza, nienawidziło ich. O ile pamiętam, ja tylko byłem neutralny.

W roku 1846 doszły nas wieści o ucieczce Ludwika Napoleona z więzienia. Rok ten był dla mnie ważny, gdyż zostałem subiektem, a nasz pryncypał, stary Jan Mincel, zakończył życie z powodów dosyć dziwnych.

W roku tym handel w naszym sklepie nieco osłabnął, już to z racji ogólnych niepokojów, już z tej, że i pryncypał za często i za głośno wymyślał na Ludwika Napoleona. Ludzie poczęli zniechęcać się do nas, a nawet ktoś (może Katz?…) wybił nam jednego dnia szybę w oknie.

Otóż wypadek ten, zamiast całkiem odstręczyć publiczność, zwabił ją do sklepu i przez tydzień mieliśmy tak duże obroty jak nigdy; aż zazdrościli nam sąsiedzi. Po tygodniu jednakże sztuczny ruch na nowo osłabnął i znowu były w sklepie pustki.

Pewnego wieczora w czasie nieobecności pryncypała, co już stanowiło fakt niezwykły, wpadł nam drugi kamień do sklepu. Przestraszeni Minclowie pobiegli na górę i szukali stryja, Katz poleciał na ulicę szukać sprawcy zniszczenia, a wtem ukazało się dwu policjantów ciągnących… Proszę zgadnąć kogo?… Ani mniej, ani więcej — tylko naszego pryncypała oskarżając go, że to on wybił szybę teraz, a zapewne i poprzednio…

Na próżno staruszek wypierał się: nie tylko bowiem widziano jego zamach, ale jeszcze znaleziono przy nim kamień… Poszedł też nieborak do ratusza.

Sprawa po wielu tłomaczeniach i wyjaśnieniach naturalnie zatarła się; ale stary od tej chwili zupełnie stracił humor i począł chudnąć. Pewnego zaś dnia usiadłszy na swym fotelu pod oknem już nie podniósł się z niego. Umarł oparty brodą na księdze handlowej, trzymając w ręce sznurek, którym poruszał kozaka.

Przez kilka lat po śmierci stryja synowcy prowadzili wspólnie sklep na Podwalu i dopiero około 1850 roku podzielili się w ten sposób, że Franc został na miejscu z towarami kolonialnymi, a Jan z galanterią i mydłem przeniósł się na Krakowskie, do lokalu, który zajmujemy obecnie. W kilka lat później Jan ożenił się z piękną Małgorzatą Pfeifer, ona zaś (niech spoczywa w spokoju) zostawszy wdową oddała rękę swoją Stasiowi Wokulskiemu, który tym sposobem odziedziczył interes prowadzony przez dwa pokolenia Minclów.

Matka naszego pryncypała żyła jeszcze długi czas; kiedy w roku 1853 wróciłem z zagranicy, zastałem ją w najlepszym zdrowiu. Zawsze schodziła rano do sklepu i zawsze mówiła:

— Gut Morgen meine Kinder! Der Kaffee ist schon fertig…

Tylko głos jej z roku na rok przyciszał się, dopóki wreszcie nie umilknął na wieki.

Za moich czasów pryncypał był ojcem i nauczycielem praktykantów i najczujniejszym sługą sklepu; jego matka lub żona były gospodyniami, a wszyscy członkowie rodziny pracownikami. Dziś pryncypał bierze tylko dochody z handlu, najczęściej nie zna go i najwięcej troszczy się o to, ażeby jego dzieci nie zostały kupcami. Nie mówię tu o Stasiu Wokulskim, który ma szersze zamiary, tylko myślę w ogólności, że kupiec powinien siedzieć w sklepie i wyrabiać sobie ludzi, jeżeli chce mieć porządnych.

Słychać, że Andrassy zażądał sześćdziesięciu milionów guldenów na nieprzewidziane wydatki. Więc i Austria zbroi się, a tymczasem Staś pisze mi, że — nie będzie wojny. Ponieważ nigdy nie był fanfaronem, więc chyba być musi bardzo wtajemniczonym w politykę; a w takim razie siedzi w Bułgarii nie przez miłość do handlu.

Ciekawym, co on zrobi! Ciekawym!…

\end{document}
