\documentclass[a4paper,12pt]{article}
\usepackage{latexsym}
\usepackage{amsmath, amssymb}
\usepackage{graphicx, float}
\usepackage{polski}
\usepackage[utf8]{inputenc}

\title{Równania różniczkowe liniowe drugiego rzędu}
\author{Zachariasz Jażdżewski}
\frenchspacing
\setlength{\parindent}{0pt}

\begin{document}
\maketitle
%\tableofcontents

\section{Definicja}
Równanie różniczkowe drugiego rzędu, które można zapisać w postaci

\[ y'' + p(x) \cdot y' + q(x) \cdot y = f(x) \]

nazywamy równaniem liniowym. \\

Funkcje $p(x), \space q(x)$ nazywamy współczynnikami, a funkcję $f(x)$ nazywamy 
wyrazem wolnym tego równania 

\begin{itemize}
    \item Jeżeli wyraz wolny jest tożsamościowo równy zero $f(x) \equiv 0$, to równanie nazywamy równaniem jednorodnym
    \item W przeciwnym przypadku nazywamy je równaniem niejednorodnym
\end{itemize}

\section{Istnienie i jednoznaczność rozwiązań}

Jeżeli funkcje $p(x), \ q(x)$ i $f(x)$ są ciągłe na przedziale $(a,b)$ oraz jeżeli $x_0 \in (a,b)$ oraz $y_0,  y_1 \in \mathbb{R}$, to zagadnienie początkowe

$$
\begin{cases}
   y'' + p(x) \cdot y' + q(x) \cdot y = f(x) \\
   y(x_0) = y_0 \\
   y'(x_0) = y_1
\end{cases}
$$

ma dokładnie jedno rozwiązanie określone na przedziale $(a,b)$

\section{Rozwiązania równania różniczkowego liniowego drugiego rzędu}

\subsection{Fakt}

Jeżeli funkcje $\phi(x)$ i $\psi(x)$ są na pewnym przedziale rozwiązaniami równania liniowego jednorodnego, to ich dowolna kombinacja liniowa

$$
y(x) = \alpha\cdot \phi(x) + \beta\cdot \psi(x)
$$

jest również rozwiązaniem równania

\section{Układ fundamentalny}

Parę rozwiązań $y_1(x), \space y_2(x)$ równania liniowego jednorodnego, określonych na przedziale $(a,b)$ nazywamy układem fundamentalnym równania na tym przedziale, jeżeli Wronskian funkcji $y_1(x)$ i $y_2(x)$ jest różny od zera

$$
W(y_1, y_2) = \det
\begin{bmatrix}
   y_1(x) & y_2(x) \\
   y_1'(x) & y_2'(x)
\end{bmatrix}
\ne 0
$$

\subsection{Fakt}

Jeżeli funkcje $y_1(x), \space y_2(x)$ tworzą układ fundamentalny równania liniowego jednorodnego, wtedy dla każdego rozwiązania $y(x)$ tego równania istnieją jednoznacznie określone stałe rzeczywiste $C_1, C_2$ takie, że

$$
y(x) = C_1 \cdot y_1(x) + C_2 \cdot y_2(x)
$$

\subsection{Rozwiązanie ogólne równania jednorodnego}

Liniową kombinację funkcji układu fundamentalnego, podaną powyżej, nazywamy rozwiązaniem ogólnym (całką ogólną) równania jednorodnego

\section{Równanie jednorodne o stałych współczynnikach}

Równanie jednorodne o stałych współczynnikach jest postaci

$$
ay'' + by' +cy = 0
$$

\subsection{Równanie charakterystyczne}

Równanie postaci

$$
ar^2 + br + c = 0
$$

nazywamy równaniem charakterystycznym równania różniczkowego

\subsection{Wielomian charakterystyczny}

Wielomian postaci

$$
w(r) = ar^2 + br + c
$$

nazywamy wielomianem charakterystycznym tego równania, a jego pierwiastki nazywamy pierwiastkami charakterystycznymi

\section{Równanie niejednorodne}

\subsection{Rozwiązanie ogólne}

Rozwiązaniem ogólnym równania niejednorodnego nazywamy sumę

$$
y(x) = C_1 \cdot y_1(x) + C_2 \cdot y_2(x) + \phi(x)
$$

$$
(\text{CORN} = \text{CORJ} + \text{CSRN})
$$

gdzie $y_1(x), \space y_2(x)$ tworzą układ fundamentalny równania jednorodnego, a $\phi(x)$ jest dowolnym rozwiązaniem równania niejednorodnego

\end{document}