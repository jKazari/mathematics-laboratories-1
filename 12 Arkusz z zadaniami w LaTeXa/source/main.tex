\documentclass[a4paper,10pt]{article}

% Math essential packages
\usepackage{latexsym, mathtools}
\usepackage{amsmath, amssymb, amsfonts, amsthm, amsxtra}
\usepackage[nomathsymbols]{polski}

% Diagrams packages
\usepackage{amscd, tikz-cd}

% Package for defining indents and paragraph skips
\usepackage[skip=10pt, indent=0pt]{parskip}

% Package for defining document size, margins etc.
\usepackage[a4paper, left=30mm, right=30mm, top=25mm, bottom=25mm]{geometry}

% Package for handling pictures, images etc.
\usepackage{graphicx, float}

% Package for coloring text
\usepackage{xcolor}

% My favourite font
%\usepackage{XCharter}

% Package for relative font resizing 
\usepackage{relsize}

% Package for creating nice headers and footers
\usepackage{fancyhdr}

% Packages for creating hyperlinks
\usepackage{url}
\usepackage[colorlinks=true,citecolor=blue,urlcolor=blue,linkcolor=blue,pdfpagemode=UseNone]{hyperref}
% Number sets
\newcommand{\N}{\mathbb{N}}
\newcommand{\Z}{\mathbb{Z}}
\newcommand{\Q}{\mathbb{Q}}
\newcommand{\R}{\mathbb{R}}
\newcommand{\C}{\mathbb{C}}

\title{Laboratorium 12}
\author{Zachariasz Jażdżewski}
\date{}

\newtheorem{zadanie}{Zadanie}
\newtheorem{definicja}{Definicja}
\newtheorem{twierdzenie}[definicja]{Twierdzenie}

%-------------------------------------------------

\begin{document}
\maketitle

\textbf{Zadanie 0.} (5 pkt) Stosując pakiet geometry ustawić marginesy: lewy 30 mm, prawy 30 mm, górny 25 mm, dolny 25 mm.

\textbf{Zadanie 1.} (20 pkt) Obliczyć podane całki:
\begin{equation*}
	\text{a)} \int x^2 \arctan x \ dx;
	\quad
	\text{b)} \int \arcsin x \ dx;
	\quad
	\text{c)} \int_{-2}^{2} x^2 + x - 3 \ dx;
	\quad
	\text{d)} \int \frac{e^{3x} - 1}{e^x - 1} \ dx;
\end{equation*}

\textbf{Zadanie 2.} (15 pkt) Zapis klamrowy funkcji
\[
	\int |x^2 - x| \ dx = 
	\begin{cases}
		\frac{x^3}{3} - \frac{x^2}{2} + C & \text{dla } x \in (-\infty, 0], \\
		\frac{x^2}{2} - \frac{x^3}{3} + C & \text{dla } x \in [0,1], \\
		\frac{x^3}{3} - \frac{x^2}{2} + \frac{1}{3} + C & \text{dla } x \in [1,\infty).
	\end{cases}
\]

\textbf{Zadanie 3.} (30 pkt) Stosując środowiska do definicji, twierdzeń, dowodów sformułować:
\begin{definicja}
	Dodawanie macierzy $A$ i $B$ określamy następująco
	\begin{equation}
		A + B = 
		\begin{pmatrix}
			a_{11} & a_{12} \\
			a_{21} & a_{22}	
		\end{pmatrix}
		+
		\begin{pmatrix}
			b_{11} & b_{12} \\
			b_{21} & b_{22}	
		\end{pmatrix}
		=
		\begin{pmatrix}
			a_{11} + b_{11} & a_{12} + b_{12} \\
			a_{21} + b_{21} & a_{22} + b_{22}
		\end{pmatrix}
		.
	\end{equation}
\end{definicja}

\begin{definicja}
	Mnożenie przez skalar $t \in \R$ definiujemy wzorem
	\begin{equation}
		tA = t 
		\begin{pmatrix}
			a_{11} & a_{12} \\
			a_{21} & a_{22}
		\end{pmatrix}
		=
		\begin{pmatrix}
			ta_{11} & ta_{12} \\
			ta_{21} & ta_{22}	
		\end{pmatrix}
	\end{equation}
\end{definicja}

\begin{twierdzenie}
	Macierz $A$ posiada macierz odwrotną $A^{-1}$ wtedy i tylko wtedy, gdy $\det A \ne 0$.
	\begin{proof}
		Pokażemy implikację w jedną stronę $\Rightarrow$, a drugą zostawimy jako ćwiczenie. Jeśli macierz jest odwracalna, to $A \times A^{-1} = I$, więc 
		\[
			\det A \times \det A^{-1} = \det (A \times A^{-1}) = \det I = 1,	
		\]
		stąd $\det A \ne 0$ 
	\end{proof}
\end{twierdzenie}

\textbf{Zadanie 4.} (30 pkt) 
\begin{enumerate}
	\item Równanie transportu
	\[
		\partial_t u + \underset{i=1}{\overset{n}{\ensuremath \sum}} b_i \frac{\partial u}{\partial x_i} = 0.
	\]
	\item Równanie Laplace'a i Poissona
	\[
		\Delta u = 0, \quad \Delta u = f(x), \quad \text{gdzie } \Delta u = \underset{i=1}{\overset{n}{\ensuremath \sum}} \frac{\partial^2u}{\partial x_i^2}.
	\]
	\item Równanie ciepła
	\[
		\frac{\partial u}{\partial t} = \Delta u.
	\]
	
\end{enumerate}

\end{document}

